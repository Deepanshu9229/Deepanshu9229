\documentclass[12pt]{article}



\title{\textbf{ Emerging Technologies in Healthcare }\\(Assignment-5)}
\date{24th February 2022\\ROll no.- A018\\SEMESTER-1\\NAME-Deepanshu patel\\BRANCH-BioMedical Engineering}

\usepackage{setspace}
\usepackage{graphicx}
\graphicspath{{images/}}

\begin{document}

\begin{figure}[h]
\centering
\includegraphics[scale=0.5]{NITRR.jpg}

\maketitle
\end{figure}



\clearpage



\textbf{1) Tricoders:-}

  A medical tricorder is a handheld portable scanning device that allows consumers to self-diagnose medical conditions and take basic vital measurements in seconds. This is not yet fully formed, but inventors are working to create and improve such a device.
  
  Actually, Wah Chang created a tricorder as a science fiction prop for the Star Trek television series. The multifunction hand-held device in the storey universe performs sensor environment scans, data recording, and data analysis—hence the term "tricorder" to refer to the three functions of sensing, recording, and computing.

 It is used to take non-invasive health measurements such as blood pressure, temperature, and blood flow. After analysing the data, it would diagnose a person's health, either as a standalone device or as a connection to medical databases via an Internet connection.

It can check many vital parameters without any physician or doctor. Imagine how useful is this, if you able to check your vital parameters anywhere and any-time.



\begin{figure}[h]
\centering
\includegraphics[scale=0.6]{tricoder.jpg}
\end{figure}


\clearpage

  \textbf{2) A LAB ON A CHIP:-} If getting samples to the lab is taking too long, why not bring the lab to the samples? That was the idea of Stanford University researchers, who recently developed "a lab on a chip" based on the CRISPR enzyme Cas12.
  
   A lab-on-a-chip is a miniaturized device that integrates into a single chip one or several analyses, which are usually done in a laboratory; It's about half the size of a credit card, has a complex network of channels smaller than the width of a human hair, and can deliver the results of a coronavirus test in under 30 minutes. Analyses such as DNA sequencing or biochemical detection can also be done. Research on lab-on-a-chip focuses on several applications including human diagnostics, DNA analysis etc.

It is very beneficial Emerging Technology beacuse of many reasons such as:-

 1.Ease of use and compactness.
 
 2.Reduction of human error.
 
 3.Faster response time and diagnosis.
 
 4.many samples store and taken in shot time.
 
 5.low cost.
 
 6.Real time process and monitoring.
 
 


\begin{figure}[h]
\centering
\includegraphics[scale=0.5]{loac.png}
\end{figure}




\end{document}