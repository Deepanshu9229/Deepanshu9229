\documentclass[12pt]{article}



\title{\textbf{ Summary of - An Introduction to the Human Body }\\(Assignment 1)}
\date{15th September 2022\\NAME-Deepanshu patel\\ROll no.- 21111018\\BRANCH-BioMedical Engineering\\SEMESTER-3}

\usepackage{setspace}
\usepackage{graphicx}
\graphicspath{{images/}}

\begin{document}

\begin{figure}[h]
\centering
\includegraphics[scale=0.5]{NITRR.jpg}

\maketitle
\end{figure}

\clearpage

\textbf{\large\textbf{Anatomy and Physiology}}
\\
\\
Anatomy is the science of body structures and the relationships among structures and physiology means the science of body functions, how it work ?
To study Anatomy and Physiology dissection is done. Dissection is the careful cutting apart of body structures to study their relationships. Some branches of anatomy are embryology, developmental biology, cell biology, histology, gross anatomy, systemic anatomy, regional anatomy, surface anatomy, radiographic anatomy, and pathological anatomy. Some branches of physiology are molecular physiology, neurophysiology,
endocrinology, cardiovascular physiology, immunology, respiratory physiology, renal physiology, exercise physiology, and pathophysiology.
\\
\\


\textbf{\large\textbf{Levels of Structural Organization and Body Systems}}
\\
\\
The human body consists of six levels of structural organization: chemical,
cellular, tissue, organ, system, and organismal.\\

\textbf{1. Cells} : cells are the basic structural and functional living units of an organism and
are the smallest living units in the human body.

\textbf{2. Tissues} : Tissues are groups of cells and the materials surrounding them that work
together to perform a particular function.

\textbf{3.Organ} : Organs are composed of two or more diff erent types of tissues; they have
specific functions and usually have recognizable shapes.

\textbf{4. Systems} : System consist of related organs that have a common function.

\textbf{5. Organism} :  An organism is any living individual.\\

 The 11 systems of the human organism: the integumentary,
skeletal, muscular, nervous, endocrine, cardiovascular, lymphatic,
respiratory, digestive, urinary, and reproductive systems.\\

\clearpage
\textbf{
\large\textbf{Homeostasis}} \\
\\
 Homeostasis is the maintenance of relatively stable conditions in the body’s
internal environment produced by the interplay of all of the body’s regulatory
processes.
When disruption of homeostasis is mild and temporary, responses of body cells quickly restore balance to the internal environment.
 Most often, the nervous and endocrine systems acting together or separately
regulate homeostasis. The nervous system detects body changes and
sends nerve impulses to counteract changes in controlled conditions. The
endocrine system regulates homeostasis by secreting hormones.
Receptors monitor changes in a controlled condition and send input to a control center. The control center sets the value (set point) at which a controlled condition should be maintained, evaluates the input it receives from receptors, and generates output commands. Effectors receive output from the control center and produce a response (effect) that alters the controlled condition.
If a response reverses the original stimulus, the system is operating by
negative feedback. If a response enhances the original stimulus, the system is
operating by positive feedback.\\
\\
\textbf{\textbf{Aging and Homeostasis}}
\\
 Aging produces observable changes in structure and function and increases
vulnerability to stress and disease. Changes associated with aging occur in all body systems.\\
\\
\textbf{\textbf{Medical Imaging}}\\ Medical imaging refers to techniques and procedures used to create images
of the human body. They allow visualization of internal structures to diagnose
abnormal anatomy and deviations from normal physiology











\end{document}