\documentclass[12pt]{article}



\title{\textbf{ Role of Sound In BioMedical }\\(Term Project)}
\date{7th April 2022\\NAME-Deepanshu patel\\ROll no.- 21111018\\BRANCH-BioMedical Engineering\\SEMESTER-1}

\usepackage{setspace}
\usepackage{graphicx}
\graphicspath{{images/}}

\begin{document}

\begin{figure}[h]
\centering
\includegraphics[scale=0.5]{NITRR.jpg}

\maketitle
\end{figure}

\clearpage
{\Large \textbf{ACKNOWLEDGEMENT}}\\
\\
\begin{large}
I am grateful to Dr.Saurabh Gupta sir for his proficient supervision of the term project on "role of sound in biomedical". I am very thankful to you sir for your guidance and support.
\\
\\
\\
\\
\\
Under the supervision of :\\
Dr.Saurabh Gupta\\
Assistant Professor\\
Bio-Medical Department

\clearpage

{\Large \textbf{ABSTRACT }}\\
\\
This report is about, Importance of sound in Biomedical or how sound derived innovations are helping our society. For evolution it is very important that we find better and convenient solution for our health and medication, there are many field through which innovation can be done. But this report is mainly focused on what is role of sound in Health or overall Biomedical field. \\
In this we see how sound can treat cancer and how sound can detect disease inside our body. As ultrasound technique are non invasive they are more convenient for patient. Acoustic levitation through sound is very interesting innovation that is also discussed in this report.
\\

\clearpage
{\Large \textbf{INTRODUCTION }}\\
\\
\begin{large}
The use of sound in medicine began long  ago. Since the early 19th century, doctors have used stethoscopes to hear the internal sounds of the human body.
\\ Sound waves help assess potentially dangerous atherosclerotic plaques, monitor chronic liver disease, and  deliver medicines to specific parts of the body. Ultrasound devices can image deep tumors in the body and concentrate sound energy on those tumors to treat the cancer. Acoustics is also merging with other disciplines such as psychology and neuroscience to  improve communication for people with speech  and hearing disabilities. Scientists are now finding  more advanced uses of sound in medicine.\\
\\
\\



\begin{figure}[h]
\centering
\includegraphics[scale=1.2]{sound1}
\end{figure}



\end{large}
\clearpage
{\LARGE \textbf{Uses of Sound in Biomedical }}\\
\\
{\large \textbf{1) Minimal Invasive Technology Offers New Hope for People with Cancer} }\\

 {\large New technology for destroying tumors by heat ablation may bring new hope for people with the most severe cancers. Developed by a research team at the French National Institute of Health Sciences (INSERM), this new technology uses an interstitial ultrasound applicator that allows efficient heating by adhering to the target tissue that destroys the ultrasound source. I am. With these probes, physicians can reach deep tumors, and  minimally invasive methods make this treatment  more affordable and better.}\\
 \begin{figure}[h]
\centering
\includegraphics[scale=0.8]{sound2}
\end{figure}
 \\
 \\
 {\large \textbf{2) Brain-Computer Interface May Help Human to Communicate Well}}\\
 \\
 {\large  With Collaborating Philip Kennedy at Neural Signals Inc. in Georgia, Boston University's Frank Guenther is inventing a brain-computer interface that records brain signals from a person's speech motor cortex and transmits them across the scalp to a computer. This computer then decodes these signals into commands for a speech synthesizer, allowing that person to hear what he/she was trying to say in real-time. With more development in this system human can communicate with other even if he/she losses his/her voice in any accident.}\\
 \\

{\large \textbf{3) Acoustic radiative impulse imaging}}\\
\\
{\large Acoustic radiative impulse imaging, as officially known, can compare images obtained using this method with traditional ultrasound images to provide additional information and often improve contrast. Therefore, it provides useful complement to traditional ultrasound. Clinical trials applying these techniques to various organs such as the liver, prostate, breast, and heart have demonstrated the usefulness of this tool, and researchers are now beginning to apply it to detect liver disease.} \\
\\
\clearpage
{\Large \textbf{RESEARCH or INNOVATION}}\\

{\large \textbf{1) Scientists were able to control monkey behaviour by zapping their brains with ultrasound waves.}}\\
\\
The University of Utah and Stanford University develop a non-invasive device similar to an ultrasound wand used for medical examinations to direct inaudible, high frequency sound waves into the brains of two monkeys in a lab. Each monkey was placed in front of a screen. The animals were first shown a target in the centre of the screen, followed by targets on the left and right sides that appeared one after the other in rapid succession. The monkeys were most likely to glance at the first target that appeared, according to the researchers. When the scientists delivered low-intensity ultrasonic waves to a part of the frontal cortex that governs eye movement, they were able to influence the animals' behaviour. The waves remotely stimulated specific neurons that control these movements, allowing the researchers to make the monkeys look left or right.
\\
\begin{figure}[h]
\centering
\includegraphics[scale=0.6]{braincontrol.jpg}
\end{figure}
\\

{\large \textbf{2) Acoustic Levitation through sound}}\\

{\large Acoustic levitation is a technology that uses acoustic radiation pressure from high-intensity sound waves to suspend items in air against gravity. It works on the same principles as acoustic tweezers, relying on acoustic radiation pressures. They offer a significant potential for lab-on-a-chip devices, which can perform laboratory operations on a microscale, such as drug development and diagnostic assays. These equipment are used to manipulate organoids, which are cell aggregates that create a more realistic milieu for biomedical research than a petri dish.\\
A group of researchers is attempting to create an acoustic device that uses sound waves to break cells, which could someday help medical professionals during diagnostics and tests. Because the sound waves are employed at such a high frequency, they cannot be heard, but the work that the acoustic device is capable of may lead to future discoveries. Moving forward, the research team hopes to improve the efficiency of its acoustic gadget, which will aid doctors in diagnosing and analysing cancer and neurological diseases}
\begin{figure}[h]
\centering
\includegraphics[scale=0.7]{acoustic.jpg}
\end{figure}
 
 \clearpage
 
 {\large \textbf{CONCLUSION}}\\
 \\
 {\large A team delivered ultrasonic waves into the brain of monkeys deciding whether to look left or right. With the right frequencies and targeting the right neurons, this researchers were able to control whether the subjects select right or left. This result takes us a step closer to able to modulate, non-invasively and reversibly, neuronal activity in specific brain circuits. This could open the way to future systematic studies of brain function in humans and to targeted personalized treatments of brain disorders. The ability to influence choice behavior non-invasively without using drugs could provide new ways to diagnose and treat disorders. \\
 
 Sound waves could be the future of biomedical research, diagnosing and treatment, says Peng Li, a chemistry professor at West Virginia University. Sound is very powerful tool to use. So it should be used for betterment of Biomedical field. It is not first time that we using  sound derived technology, we are using stethoscope, ultrasound scan, hearing aid etc from many years. Overall, now it is time to feel the importance of sound and do more and more exciting innovations.}\\
 \\
 \clearpage
 
\begin{large}
\textbf{ REFERENCE }\\
\\
following are the websites from where ideas are taken:-
\\
\\
1)https://en.wikipedia.org\\
\\
2)aip.scitation.org\\
\\
3)www.eurekalert.org\\
\\
4)www.ncbi.nlm.nih.gov\\
\\
5)www.news-medical.net\\
\\
6)www.google.com\\
\end{large}




\end{large}




\end{document}

